\subsection{The Virginia Model} \label{virginia_model}
We build a statistical model to identify features that contribute to a household's decision to adopt rooftop solar panels.
However, due to low penetration of solar adopters in rural regions, the data on
solar adopters is sparse, which makes it difficult to build a good
prediction model. Given highly imbalanced training data, traditional statistical methods
tend to predict most households to be non-adopters in order to minimize
the mis-classification error and provide high overall prediction
accuracy.

In our study, we are more interested in identifying potential adopters
so we apply a decision-adjusted modeling approach 
from Mao et al.~\cite{Mao}, and Hu et al.~\cite{hu19rooftop}. The decision-adjusted
approach optimizes the prediction model with respect to a user-stated
evaluation criterion. We set this criteria to maximize the sum of True
Positive Rate and True Negative Rate. The decision-adjusted approach can be applied to
different statistical models, here we choose logistic regression model as
our baseline model.



\begin{table}[H]
	\centering
	\caption{This table provides the list of features used in the
          regression model.}
	\begin{tabular}{|l|p{5cm}|}
		\hline
		{\bf Feature} & {\bf Description} \\ 
		\hline
		acreage & Acreage of the house. \\ 
		\hline
		area\_type & Rural (0) or urban (1). \\ 
		\hline
		asrYear & Year house was built.  \\ 
		\hline
		baths & Num of bathrooms. \\ 
		\hline
		bedrooms & Num of bedrooms. \\ 
		\hline
		NPV & Net Present Value. \\ 
		\hline
		climateRegionPub & Type of climate. \\ 
		\hline
		dailyConsumption & Avg. electricity used in Watt
                                   hrs. \\ 
		\hline
		education & Highest education achieved. \\ 
		\hline
	    fuelheat & Type of fuel used for heat. \\ 
	    \hline
		income & Household's yearly income.  \\ 
		\hline
		numCarStorage & Num. of car storage in the house. \\
		\hline 
		sqFootage & Sq. footage of the house.  \\ 
		\hline
		swimpool & Pool present or not. \\ 
		\hline
		totalVal & Estimated house value. \\
		\hline 
		typehuq & Type of housing unit. \\ 
		\hline
		householdSize & Family size. \\ 
		\hline
		X1Mile & Adopters within 1 mile. \\ 
		X2Mile & Adopters within 2 mile. \\ 
		X3Mile & Adopters within 3 mile. \\ 
		X4Mile & Adopters within 4 mile. \\ 
		\hline
	\end{tabular}
	\label{tab: feature description}
\end{table}

The decision-adjusted solar adopter prediction model is as
follows. Suppose a prediction model is obtained from data
$\{X_i,I_i(\delta),Y_i\}_{i=1}^n$, where $X_i$ are the demographic features as
shown in Table~\ref{tab: feature description}.  $Y_i$ is the response,
$Y_i = 1$ means household is an adopter and $Y_i = 0$ means non-adopter, $I_i(\delta)$ are the indicator features for corresponding demographic features, which is defined as
\begin{equation*}
I_i(\delta)=\left\{
\begin{array}{@{}ll@{}}
1, & \text{if }X_i > \delta;\\
0, & \text{if }X_i \leq \delta,
\end{array}\right.
\end{equation*}

The indicator features are introduced when the coefficients of linear combination are not able to
capture all information in the model. For example, if the coefficient of a feature
is positive, then a larger value of the feature will increase the
likelihood of adoption. However, this positive relationship may not be
constant; it may be strong when the value of the
feature is small, and weak when the value of the feature is
large. The indicator features address this issue.

The logistic regression model has the link function below,
\begin{align*}
\text{Pr}(Y=1|X, I(\delta))=\frac{exp\Big(\big(X, I(\delta)\big)^T\beta\Big)}{1+exp\Big(\big(X, I(\delta)\big)^T\beta\Big)}
\end{align*}
Where $\beta$ is the regression coefficient. For the logistic regression with elastic net, $\beta$ is estimated through maximizing a penalized log-likelihood $l(\beta)-P(\beta,\alpha,\lambda)$, where 
\begin{align*}
P(\beta,\alpha,\lambda)=\lambda\big((1-\alpha)\frac{1}{2}||\beta||_2^2+\alpha||\beta||_1\big)
\end{align*}


The predicted label for this household is determined by the following, where $\tau$ is the threshold probability for determining whether a household is adopter or not. 
\begin{equation*}
\hat{Y_i}=\hat{Y_i}(X_i,I_i(\delta),\alpha,\lambda,\tau)=\left\{
\begin{array}{@{}ll@{}}
1, & \text{if Pr}(Y_i=1)\ge \tau \\
0, & \text{if Pr}(Y_i=1)< \tau
\end{array}\right.
\end{equation*}
Then the optimization problem for our decision-adjusted model is
\begin{equation*}
\max_{\delta,\alpha,\lambda,\tau} \Omega\Big(\hat{Y_i}(X_i,I_i(\delta),\alpha,\lambda,\tau),Y\Big)
\end{equation*}
Where $\Omega(\cdot)$ is a function of predicted label
$\hat{Y_i}(X_i,I_i(\delta),\alpha,\lambda,\tau)$, and true label
$Y$. It is determined by the specific goals of the study, as defined
by the user. 

In the decision adjusted model, we want to maximize ($\Omega=$
TPR$+$TNR) given $\tau$ and $\delta$. Here `TPR' refers to true
positive rate, `TNR' refers to true negative rate, `FP' refers to false positives, and `FN' refers to false negatives. The TPR$+$TNR is defined as:

%\begin{equation*}
%\Omega=\text{TPR}+\text{TNR}=\frac{TP}{TP+FN}+\frac{TN}{TN+FP}.
%\end{equation*}

\begin{equation*}
            \Omega=\frac{TP}{TP+FN}+\frac{TN}{TN+FP}=\frac{TP}{Condition\ positives}+\frac{TN}{Condition\ negatives},
        \end{equation*}

This is an improvement over accuracy as the performance matrix, which
is given by:

\begin{equation*}
        \text{ACCURACY}=\frac{TP+TN}{TP+FN+TN+FP}=\frac{TP+TN}{n}.
        \end{equation*}


Table \ref{tab: model coefficients} summarizes the model results. The climateRegion, education, fuelheat, type of housing unit are categorical features, the coefficient for each level are shown in the table.

The tuning paraments selected from the decision-adjusted approach are: \\
$\bullet$ $\tau = 0.0022$, a household with predicted probability greater than 0.0022 is classified as an adopter.\\
$\bullet$ $\lambda = 0.000303$, $\alpha = 0.4$, which are values of tuning parameters in enet penalty.\\
$\bullet$ $\delta(\text{totalVal}) = 75000$, if the value of feature totalVal is greater than 75000, then the value of the corresponding indicator feature is 1.\\

	
\begin{table}[H]
	\centering
	\scriptsize
    \caption{Model features and the corresponding coefficients.}
	\begin{tabular}{|p{6cm}|r|}
		\hline
		Variable & Coefficient \\ 
		\hline
		\hline
		intercept & -10.6 \\ 
		\hline
		acreage & -0.123 \\ 
		\hline
		area\_type & -1.79 \\ 
		\hline
		asrYear & 0.00184 \\
		\hline 
		baths &    0 \\ 
		\hline
		bedrooms & 0.0881 \\ 
		\hline
		dailyConsumption & 5.65e-07 \\ 
		\hline
		income &    0 \\ 
		\hline
		NPV &    0 \\ 
		\hline
		numCarStorage & 0.331 \\ 
		\hline
		sqFootage &    0 \\ 
		\hline
		swimpool & 0.169 \\ 
		\hline
		totalVal & -3.66e-07 \\  
		\hline
		householdSize & 0.123 \\ 
		\hline
		\textbf{climateRegion} & \\
		PubCold/Very Cold & -1.09 \\ 
		Hot-Humid & 0.00883 \\ 
		Mixed-Humid &    0 \\ 
		\hline
		\textbf{education} & \\
		Less than high school diploma or GED & 0 \\
		High school diploma or GED  & -0.0117 \\ 
		Some college or Associate’s degree  & 0.332 \\ 
		Bachelor’s degree  &    0 \\ 
		Masters, Professional, or Doctorate degree  &    0 \\ 
		\hline
		\textbf{fuelheat} & \\
		Natural gas from underground pipes  & 0 \\ 
		Propane (bottled gas)  & -0.528 \\ 
		Fuel oil/kerosene  & 0.269 \\ 
		Electricity  &    0 \\ 
		Wood  & 0.472 \\ 
		\hline
		\textbf{Type of housing unit} & \\
		Mobile home  & 0 \\
		Single-family detached house  & 0.00162 \\ 
		Single-family attached house  & -0.479 \\
		\hline
		\textbf{Neighborhood feature} & \\
		X1Mile & 0.399 \\ 
		X2Mile &    0 \\ 
		X3Mile & 0.0299 \\ 
		X4Mile & 0.0376 \\ 
		\hline
		Indicator feature for totalVal & -1.12 \\ 
		\hline
	\end{tabular}
	\label{tab: model coefficients}
\end{table}