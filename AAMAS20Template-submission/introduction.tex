\section{Introduction}

\begin{itemize}
    \item Complex large-scale agent-based models are becoming more common, in several application areas.
    \item These models are data-driven and specific, customized to answer specific questions or model specific phenomena.
    \item Often, we see multiple models for the same general phenomenon, such as epidemic spread, technology adoption, disaster evacuation, etc., created with different parameters, data sources, and model structure.
    \item This raises the general question of how to compare such models~\cite{axtell96aligning,burton99validation}.
    \item For example, in the case of the adoption of rooftop solar panels by households in two different parts of the country, we might wish to know whether it is somehow easier to have a large number of adoptions in one region than another, or whether it is just a chance difference.
    \item In this paper, we present a general framework to make these types of model comparisons. Then we present a specific example of the application of this framework to the comparison of two different models of the adoption of rooftop solar panels by households in two different regions of the United States.
    \item The rest of this paper is organized as follows. First we present the general methodological framework. Then we describe the two models we are comparing. In this specific case, model comparison requires learning a phase transition boundary in a contagion model. However, since the agent-based simulations are expensive (time-consuming) to run, we develop an active learning method with the simulation in the loop. After describing this method in detail, we present results from computational experiments with the two models. We end with a discussion of related work and future directions.
\end{itemize}